\documentclass[a4paper,11pt,onecolumn,oneside,UTF8]{article}

\newcommand*{\dif}{\mathop{}\!\mathrm{d}}
\usepackage{ctex}     % 中文支持
\usepackage{amsthm,amsmath,amssymb}
\usepackage{mathrsfs}
\usepackage{bm}       % 公式中的粗体字符(用命令\boldsymbol)
\usepackage{graphicx, subfigure}
\usepackage{caption}
\usepackage{float}
\usepackage{hyperref}
\usepackage{multirow}
\usepackage{color}
\usepackage{listings}
\usepackage{graphicx}
\usepackage{pythonhighlight}
\usepackage{hyperref}
\addtolength{\topmargin}{-54pt}
\setlength{\oddsidemargin}{-0.9cm}  % 3.17cm - 1 inch
\setlength{\evensidemargin}{\oddsidemargin}
\setlength{\textwidth}{17.00cm}
\setlength{\textheight}{24.00cm}    % 24.62

\begin{document}

\begin{center}
    \Huge\textbf{weight-normalization VS threshold-balancing}\footnote[1]{*:下面探讨针对的都是软重置的脉冲发放方式}
    \\[20pt]
    \Large\textnormal{2022 年4 月}
\end{center}

\section*{符号表示}
$l$: $l \in \{1, \cdots ,L\}$ a network with L layers, layer $l$

$W_{ij}^l$: weight connection between neuron $i$ in layer $l$ and neuron $j$ in layer $l-1$

$b_i^l$: neuron $i$ bias in layer $l$

$a_i^l$: neuron $i$ activation value(after relu) in layer $l$

$M^l$: the number of neurons in layer $l$

$V_i^l(t)$: neuron $i$ membrane potential in layer $l$ and time $t$

$z_i^l(t)$: neuron $i$ inputs in layer $l$ and time $t$

$\Theta_{t,i}^{l}$: a step function indicating the occurrence of a spike at time t

$N_i^l(t)$: the number of spike generated in neuron $i$ layer $l$ for a total time t

$r_i^l(t)$: firing rate of neuron $i$ in layer $l$ for a total time t

$V_{\text {thr }}^l$: voltage threshold in layer $l$


\section*{满足关系}
$$\Theta_{t, i}^{l}=\Theta\left(V_{i}^{l}(t-1)+z_{i}^{l}(t)-V_{\text {thr }}\right), with \; \Theta(x)= \begin{cases}1 & \text { if } x \geq 0 \\ 0 & \text { else }\end{cases}$$


$$N_i^l(t) = \sum_{t'=1}^t\Theta_{t',i}^l$$

$$r_i^l(t) = N_i^l(t)/t$$

$$a_i^l = \sum_{j=1}^{M^{l-1}}W_{ij}^la_j^{l-1} + b_i^l$$

$$V_i^l(t)=V_{i}^{l}(t-1)+z_{i}^{l}(t)-V_{\text {thr }}\Theta_{t,i}^{l}$$

\begin{equation}
    z_{i}^{l}(t) = V_{\text {thr }}\left(\sum_j^{M^{l-1}}W_{ij}^l\Theta_{t,i}^{l}+b_i^l\right) \tag*{*}
\end{equation}



\section*{对第一层}
\textbf{a)权重归一化:}
$$V_i^l(t)=V_{i}^{l}(t-1)+\underbrace{\sum_j^{M^{l-1}}W_{ij}^lx_j+b_i^l}_{a_i^l}-\Theta_{t,i}^{l}$$

移项并对T个时间步长求和:
$$\sum_{t=0}^T\Theta_{t,i}^{l}=\sum_{t=0}^T \sum_j^{M^{l-1}}W_{ij}^lx_j+b_i^l - V_i^l(T)$$

由于输入的x是固定值, 所以第一层的Wx+b是常数项, 式子变成
$$\sum_{t=0}^T\Theta_{t,i}^{l}=\left(\sum_j^{M^{l-1}}W_{ij}^lx_j+b_i^l\right)T - V_i^l(T)$$

两边除以$T$
\begin{equation}
    r_i^l(T) = \underbrace{\sum_j^{M^{l-1}}W_{ij}^lx_j+b_i^l}_{a_i^l} - \frac{V_i^l(T)}{T}
\end{equation}

Diehl当时对权重归一化的解释为:转换后由于阈值和权重的比例不合适,导致神经元欠激活或者过激活,提出权重归一化来减少这个误差。
实际上由式子(1),\textbf{另一种解释方法为:}要想RELU单元用firing rate approximation of an IF neuron with no refractory period近似, 这里的firing rate即$r_i^l(T) \in [0, 1]$,需要将模拟激活值$a_i^l$映射到[0,1],才能用不带不应期的IF神经元脉冲发放率做近似。这就是为什么除以激活值最大值,即权重归一化的数学解释

同时只有对正的输入,膜电势才会积累到超过阈值发放脉冲,这就使得负的输入膜电势越来越负,脉冲发放率为零,完美地用IF神经元等价了RELU激活函数。

\textbf{所以注意(1)式子中的W, b是归一化后的值},若是换一种表达,即如果W, b是ANN的原始权重, 那么(1)式子应该写为:

\begin{equation}
    r_i^l(T) = \underbrace{\sum_j^{M^{l-1}}\frac{W_{ij}^l}{a_{max}^l}x_j+\frac{b_i^l}{a_{max}^l}}_{a_i^l/a_{max}^l} - \frac{V_i^l(T)}{T}
\end{equation}


\textbf{b)阈值平衡:}

上周讨论了发放脉冲是1还是$V_{\text {thr }}$的问题, 显然发放$V_{\text {thr }}$从生物上不符合脉冲的特性,\textbf{但是下面我们先说明: 发放$V_{\text {thr }}$和权重归一化等价}。这时把$V_{\text {thr }}\Theta_{t,i}^{l}$看做一个脉冲数
\begin{equation}
    V_i^l(t)=V_{i}^{l}(t-1)+\underbrace{\sum_j^{M^{l-1}}W_{ij}^lx_j+b_i^l}_{a_i^l}-V_{\text {thr }}\Theta_{t,i}^{l}
\end{equation}

移项并对T个时间步长求和:
$$V_{\text {thr }}\sum_{t=0}^T\Theta_{t,i}^{l}=\sum_{t=0}^T \sum_j^{M^{l-1}}W_{ij}^lx_j+b_i^l - V_i^l(T)$$

由于输入的x是固定值, 所以第一层的Wx+b是常数项, 式子变成

$$V_{\text {thr }}\sum_{t=0}^T\Theta_{t,i}^{l}=\left(\sum_j^{M^{l-1}}W_{ij}^lx_j+b_i^l\right)T - V_i^l(T)$$

两边除以$T$,由于$r_i^l(T) = \frac{N_i^l(T)}{T} = \frac{V_{\text {thr }}\Theta_{t,i}^{l}}{T}$
\begin{equation}
    r_i^l(T) = \underbrace{\sum_j^{M^{l-1}}W_{ij}^lx_j+b_i^l}_{a_i^l} - \frac{V_i^l(T)}{T}
\end{equation}

这里的$V_i^l(T)$是(2)式中的$V_{\text {thr }}$倍,在$V_{\text {thr }}$设为$a_{max}^l$的情况下,显然权重归一化后的脉冲发放率乘以最大值后,等于阈值平衡的脉冲发放率。



----------以下先忽略 ---------------

注意\textbf{满足关系}一节中的(*)式, 它的输入很奇怪, 是输入又乘以了一个电压阈值。\textbf{在发放脉冲为1的情况下, (*)式这样的输入才能使两种不同发放的阈值平衡等价}。所以所有输入前面多乘了一项$V_{\text {thr }}$,(3)式变换为

\begin{equation}
    V_i^l(t)=V_{i}^{l}(t-1)+V_{\text {thr}}\underbrace{\sum_j^{M^{l-1}}W_{ij}^lx_j+b_i^l}_{a_i^l}-V_{\text {thr }}\Theta_{t,i}^{l}
\end{equation}
由于$r_i^l(T) = \frac{N_i^l(T)}{T} = \frac{\Theta_{t,i}^{l}}{T}$,可以自然得到与(4)式一样的结果


\section*{对更高层}
$$
V_i^l(t)=V_{i}^{l}(t-1)+\sum_j^{M^{l-1}}W_{ij}^l\Theta_{t,j}^{l-1}+b_i^l-V_{\text {thr }}\Theta_{t,i}^{l}
$$

同样地,进行T次求和:
$$V_{\text {thr }}\sum_{t=0}^T\Theta_{t,i}^{l}=\sum_{t=0}^T\left(\sum_j^{M^{l-1}}W_{ij}^l\Theta_{t,j}^{l-1}+b_i^l\right) - V_i^l(T)$$

两边同时除以T
\begin{equation}
    r_i^l(T) = \sum_j^{M^{l-1}}W_{ij}^lr_j^{l-1}+b_i^l - \frac{V_i^l(T)}{T}
\end{equation}

这是阈值平衡的,显然权重归一化后的脉冲发放率乘以最大值后,仍然等于阈值平衡的脉冲发放率,即两者在高层也完全等价。

\textbf{激活值与脉冲发放率之间的关系}

由(6)式可得不同层间的脉冲发放率关系。为了方便表示,我们用权重归一化的形式进行展开。

$$
\begin{aligned}
    r_i^l(T) &=\sum_j^{M^{l-1}}\frac{W_{ij}^la_{max}^{l-1}}{a_{max}^l}r_j^{l-1}+\frac{b_i^l}{a_{max}^l} - \frac{V_i^l(T)}{T} \\
    &=\sum_j^{l-1} \sum_j^{M^{l-1}}\frac{W_{ij}^la_{max}^{l-1}}{a_{max}^l}\left(\sum_k^{M^{l-2}}\frac{W_{jk}^{l-1}a_{max}^{l-2}}{a_{max}^{l-1}}r_k^{l-2}+\frac{b_j^{l-1}}{a_{max}^{l-1}} - \frac{V_j^{l-1}(T)}{T} \right) + \frac{b_i^l}{a_{max}^l} - \frac{V_i^{l}(T)}{T}\\
    &= \begin{split}
        &\sum_j^{l-1} \sum_j^{M^{l-1}}\frac{W_{ij}^la_{max}^{l-1}}{a_{max}^l}\left(\sum_k^{M^{l-2}}\frac{W_{jk}^{l-1}a_{max}^{l-2}}{a_{max}^{l-1}}\left(\sum\cdots \underbrace{\sum_m^{M^{1}}\frac{W_{1m}^1}{a_{max}^1}x_m+\frac{b_m^1}{a_{max}^1}}_{a_m^1/a_{max}^1} - \frac{V_i^1(T)}{T} + \cdots + \frac{b}{a_{max}} - \frac{V_i^l(T)}{T} \right) \right.\\ 
        &\left. + \frac{b_j^{l-1}}{a_{max}^{l-1}} - \frac{V_j^{l-1}(T)}{T} \right) + \frac{b_i^l}{a_{max}^l} - \frac{V_i^{l}(T)}{T}
    \end{split}
\end{aligned}
$$

用$\Delta V_i^l$表示$\frac{V_i^{l}(T)}{T}$

则上面的式子变为
\begin{equation}
    r_i^l(T) = \frac{a_i^l}{a_{max}^l} - \Delta V_i^l - \sum_j^{M^{l-1}}\frac{W_{ij}^la_{max}^{l-1}}{a_{max}^l}\Delta V_j^l - \cdots -\sum_j^{M^{l-1}}\frac{W_{ij}^la_{max}^{l-1}}{a_{max}^l} \cdots \sum_k^{M^{1}}\frac{W_{1k}^2 a_{max}^1}{a_{max}^2} \Delta V_k^1
\end{equation}

所以从理论上来讲,用脉冲发放率*最后一层的最大激活值来近似模拟激活值$a_i^l$,有(7)式中,减号后面一长串的固有误差,这个误差每项都乘了$\Delta V$,即$\frac{V(T)}{T}$。所以增加时间步长可以有效地减少转换的损失。
对于分类问题,虽然这个误差对每个神经元都存在,但是时间步长的增大,这个误差往往不大,所以对取最大值作为分类结果这样的任务,需要的时间步长不大。但对于检测,这样的误差会引起浮点数计算的改变,所以需要更大的时间步长来减少这一固有损失。

那么既然有了固有误差,人为地将它补上即把在得到脉冲发放率后,将右边误差加到发放率上,就可以无损失地还原模拟激活值了,而且不受时间步长的影响。遗憾的是理论上确实可行,但是这完全没有生物合理性,也失去了脉冲神经网络的意义;从代码实现来讲,这些权重是使得输入$W\Theta+b$为正的权重,将每层的这些权重找出来再和$\Delta V$做乘法就过于复杂了。
所以在为了简化,转换中人们还是带着这样的误差在做视觉任务。



\end{document}